\documentclass[a4paper, 12pt, oneside]{report} % for final draft compilation
% \documentclass[a4paper, 12pt, oneside, draft]{report} % for draft compilation

% Preamble:
\usepackage[T1]{fontenc}                                 % package for specifying output font encoding
% \usepackage[font=footnotesize,format=hang]{caption}      % package for formatting figure caption
\usepackage[hidelinks, colorlinks]{hyperref}             % package for hyperlinking references - hidelinks option removes the ugly boxes around the hyperlinks
\usepackage[md]{titlesec}                                % package for tweaking the chapter/section/subsection titles format
\usepackage[utf8]{inputenc}                              % package for specifying input font encoding
\usepackage{chngcntr}                                    % package for making continuous figure numbering, instead of section-based numbering
\usepackage{graphicx}                                    % package for graphics handling (i.e. figure images)
\usepackage{mathptmx}                                    % package for Times New Roman font
\usepackage{natbib}                                      % package for reference management
\usepackage{setspace}                                    % package for setting line space (used in title page)
\usepackage{booktabs}                                    % package for better looking tables
\usepackage{etoolbox}                                    % package to 'robustify' \bfseries

% Set up the hyperlinks - change the colour of the links to blue:
% \hypersetup{allcolors=blue}
\hypersetup{allcolors=black} % uncomment this line to make the hyperlinks black

% Change the default url font:
\urlstyle{rm}

% Change the title of Table of Contents:
\renewcommand{\contentsname}{Table of Contents}

% Italicise the subsubsection title:
\titleformat{\subsubsection}{\normalfont\itshape\bfseries}{\thesubsubsection}{1em}{}
\titleformat{\paragraph}{\normalfont\itshape}{\theparagraph}{\noindent}{}

% Specify graphics path:
\graphicspath{{./images/}}
\DeclareGraphicsExtensions{.pdf,.png,.jpg}

% Remove chapter header:
\titleformat{\chapter}[display] {\normalfont}{}{0pt}{\Huge}
\titleformat{\section}[display] {\normalfont}{}{0pt}{\Large}

\begin{document}

\begin{titlepage}

\centering

\null

\vfill

\rule[2.0mm]{\textwidth}{0.5mm}

\begin{doublespace}
    {\Huge \scshape Digital Lab Notebook with Git/GitHub and Markdown}
\end{doublespace}

\rule[2.0mm]{\textwidth}{0.5mm}

{\large \scshape Instructor's Notes}

\vfill

\end{titlepage}

\pagenumbering{roman} % change the page numbering to roman numeral

% Set line spacing to double space:
\onehalfspacing

\addcontentsline{toc}{section}{\contentsname}

\tableofcontents

\clearpage

% Need to set line spacing to single space, otherwise it inserts a blank page for an unknown reason:
\singlespacing

\clearpage

% Reset the spacing to 1.5:
\onehalfspacing

\pagenumbering{arabic} % revert the page numbering to normal numbering

% Begin main text from here:

\chapter{Introduction\\($\sim$40min)}
\label{cha:introduction}

\section{What is git?}
\label{sec:what_is_git}

{\bfseries Duration: 15min}

\begin{itemize}
	\item Before you begin, make sure everyone has git installed in their local machine (for later use)
	\item Start with examples of keeping track of changes of a document (e.g.\ thesis/paper)
		\begin{itemize}
			\item MS Word keeps record of changes of the {\bfseries current} document -- when you close and re-open the document, you lose all of the changes you made previously
			\item To avoid the situation as above, people usually make a copy before they start editing, and then (eventually) compare the new version with the old one (which is a hassle)
		\end{itemize}
	\item Example FAQ: ``Can you reproduce your results?'' (were your methods/protocols well documented?)
	\item Example FAQ: ``What protocol(s) did you use for that experiment?'' (when you have multiple trial/error attempts)
	\item Make absolutely clear the objective(s) of using a version control system
		\begin{itemize}
			\item To keep track of and maintain the most current version of your document
			\item Keep a long-living history of your document/protocols
			\item Share documents with others for collaboration
			\item `Back-up' of your codes
		\end{itemize}
\end{itemize}

(Make a diagram somewhere)


\section{Creating GitHub}
\label{sec:creating_github}

{\bfseries Duration: 10min}

\begin{itemize}
	\item Create a GitHub account
	\item Explain what GitHub is
		\begin{itemize}
			\item A ``remote'' server for your git repository
			\item Essentially an online storage space for the documents that you want to version control
		\end{itemize}
	\item Create a DLN repository to work with during the course
	\item Explain what a repository is
\end{itemize}


\section{Markdown files}
\label{sec:markdown_files}

{\bfseries Duration: 15min}

\begin{itemize}
	\item Edit README.md
	\item Explain and show what a Markdown document is
	\item Introduce some Markdown syntax
		\begin{itemize}
			\item Headers
			\item Text decorations (bold, italics, underline, etc.)
			\item Web Links
			\item Adding images
			\item Footnotes
			\item Tables
		\end{itemize}
\end{itemize}

\noindent
Play around with Markdown syntax using the README file (saving changes and committing comes in the next chapter/section).


\chapter{Making Digital Lab Notebook\\($\sim$50min)}
\label{cha:making_digital_lab_notebook}

\section{Change, add, and commit}
\label{sec:change_add_and_commit}

{\bfseries Duration: 20min}

\begin{itemize}
	\item Save changes and commit the README.md
	\item Explain what committing means
	\item Show what the logs/commit history look like
	\item Explain what the git commit ID/hash are
	\item Add/remove more stuff to the README file and commit changes
	\item Note the diff of the documents when making changes
	\item Note that, althought you can ``checkout'' previous commits, you can't ``revert'' the repo to that particular commit on GitHub
		\begin{itemize}
			\item Reverting to different commit will be covered in command line git
		\end{itemize}
	\item Keep adding changes to the README and demonstrate how git/GitHub is keeping track of all the file changes
\end{itemize}


\section{Creating new entries}
\label{sec:creating_new_entries}

{\bfseries Duration: 15min}

\begin{itemize}
	\item Start creating new entries to the repo
		\begin{itemize}
			\item Make a new directory in the repo for your entries (e.g.\ protocols, notes, results)
		\end{itemize}
	\item Add an entry to the directories as before and commit
	\item Keep showing what the commits look like through the commit history
\end{itemize}


\section{Linking entries}
\label{sec:linking_entries}

{\bfseries Duration: 15min}

\begin{itemize}
	\item Start creating links to your new entries in README
		\begin{itemize}
			\item Explain directory structure when creating links to new entries
		\end{itemize}
	\item Point out the power of doing this
		\begin{itemize}
			\item README as the content viewer/navigator
			\item Individual entries organised in directories
			\item Everything version controlled with git
		\end{itemize}
\end{itemize}


\chapter{Working Collaboratively and Locally\\($\sim$50min)}
\label{cha:working_collaboratively_and_locally}

\section{Sharing your repository}
\label{sec:sharing_your_repository}

{\bfseries Duration: 15min}

\begin{itemize}
	\item Just send your collaborator the URL to your repository
	\item Demonstrate how to give editing access to others for collaboration
	\item Let everyone play around with editing other people's repository and sending pull requests
	\item Introduce the idea of merge conflicts and let them try it out
\end{itemize}



\section{Why use git on local machine?}
\label{sec:why_use_git_on_local_machine}

{\bfseries Duration: 15min}

\begin{itemize}
	\item Able to checkout, but can't revert to a particular commit on GitHub
	\item For bulk adds/commits (edit locally, and then push commits)
	\item You can't have a ``working draft'' when you are editing it on GitHub -- whatever change you make to the document, you have to commit it after each edit
		\begin{itemize}
			\item You can, but you will have to create a new branch for that (which could be a hassle to keep track of)
		\end{itemize}
\end{itemize}


\section{Setting up git on your local machine}
\label{sec:setting_up_git_on_your_local_machine}

{\bfseries Duration: 20min}

\begin{itemize}
	\item Follow first half of the SWC git lesson for git setup
	\item Clone the already created repository from GitHub
	\item Repeat similar process as what we did on GitHub (add changes, commit, log, revert, push)
	\item Explain new ideas (i.e.\ add, revert, pull, and push)
\end{itemize}


\end{document}
